\begin{abstract}
ABSTRACT
ANALYSIS OF THE ODOO SOFTWARE CAPABILITIES REGARDING PRODUCT LIFECYCLE MANAGEMENT, MANUFACTURING EXECUTION SYSTEMS AND THEIR INTEGRATION
The second half of the 20th century had been marked for the advancements of computer technology in all aspects of production. 
The key feature of that statement is the undeniable truth that alongside the increased complexity allowed by computing power comes an ever increasing production of overwhelming amounts of information.
 From separate perspectives of the industrial landscape, several systems were brewed by that sheer necessity for organization, automation and waste reduction focusing on that pool of useful data. 
ERP (from a managerial perspective), MES (from a production perspective) and more recently PLM (from a strategic development/redevelopment perspective) emerged as information solutions tackling this problem from different angles. 
These solutions, however effective, are always plagued by the fundamental incompatibility between the tools that implement those systems. This paper objectives revolve around analyzing the integration PLM and MES systems from a theoretical perspective and comment on the use of the Odoo software tool to implement said integration. 
The Odoo software was described in detail (regarding its use for manufacturing envirorment) icluding how it implements PLM and MES. Then, the software was subjected to the simulation of a fictional firm devised in the molds of Industry 4.0. This company was a fictional recently founded small case manufacturing company that uses plastic injection molding as their primary mean of production and uses additive manufacturing and fast prototyping as part of their business strategy. 
Keywords: Product Life-Cycle Management, Product Life-Cycle Management, Odoo

摘要
對 Odoo 軟件在產品生命周期管理、製造執行系統及其整合方面的能力進行分析
20世紀下半葉以來,計算機技術在生產的各個方面取得了進步。
這個陳述的關鍵特徵是不可否認的事實,即隨著計算能力所允許的複雜性增加,信息的生產量也在不斷增加。
從工業景觀的不同角度來看,幾個系統因為對組織、自動化和減少浪費的需求而被發展出來,集中於這些有用數據的範圍。
從管理角度來看是 ERP,從生產角度來看是 MES,還有最近的 PLM(從戰略開發/重建的角度來看)作為信息解決方案從不同角度解決這個問題。然而,這些解決方案總是受到實施這些系統的工具之間基本不兼容性的困擾。
本文的目標是從理論角度分析 PLM 和 MES 系統的整合,並評論使用 Odoo 軟件工具來實現該整合。
對 Odoo 軟件進行了詳細描述(關於其在製造環境中的使用),包括它如何實現 PLM 和 MES。然後,對該軟件進行了在 Industry 4.0 的模式下設計的虛構公司的模擬。該公司是一家虛構的新成立的小型案例製造公司,主要使用塑料射出成型作為其主要生產手段,並將增材製造和快速原型製造作為其業務策略的一部分。
關鍵詞:產品生命周期管理,Odoo,製造執行系統
\end{abstract}